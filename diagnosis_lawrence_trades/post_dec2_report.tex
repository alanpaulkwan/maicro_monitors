% Post-Dec 2 Tracking Error Analysis - Article Report
\documentclass[11pt]{article}
\usepackage[margin=1in]{geometry}
\usepackage{booktabs}
\usepackage{amsmath}
\usepackage{graphicx}
\usepackage{hyperref}

\title{Post–Dec 2 Tracking Error Analysis\\\large Lawrence vs Jianan Model}
\author{Maicro Monitoring}
\date{\today}

\begin{document}
\maketitle

\section{Context}

We analyze how well Lawrence's executed positions track Jianan's crypto model
for the period starting \textbf{2025-12-02} (``post--Dec 2'').
Positions are compared at the \emph{holdings date} level against the model's
target weights, aligned using the correct \textbf{T+2} convention:
signals from day $T$ should correspond to holdings on day $T+2$.

In this period there are
\[
N_{\text{post}} = 182
\]
symbol--day observations.

\section{Top-Level Waterfall}

Each symbol--day is classified into one of five mutually exclusive categories:
MATCHED, MISSING\_POSITION, WRONG\_SIGN, MAGNITUDE\_ERROR, and EXTRA\_POSITION.
Empirically, EXTRA\_POSITION is negligible in this sample.

\begin{table}[h]
  \centering
  \caption{Waterfall Breakdown (Post--Dec 2, 182 position-days)}
  \begin{tabular}{lrr}
    \toprule
    Category          & Count & Share (\%) \\
    \midrule
    MATCHED           & 40    & 21.9 \\
    MISSING\_POSITION & 84    & 45.9 \\
    MAGNITUDE\_ERROR  & 36    & 19.9 \\
    WRONG\_SIGN       & 22    & 12.3 \\
    EXTRA\_POSITION   & $\approx 0$ & $< 1$ \\
    \bottomrule
  \end{tabular}
\end{table}

Only about $22\%$ of symbol--days match the model within tolerance; the remaining
$78\%$ are missing, wrong sign, or mis-sized.

\section{Missing Positions}

For MISSING\_POSITION cases, we compute the intended notional
\[
\text{target\_notional}
  = |\text{target\_weight}| \times \text{portfolio\_value}
\]
and compare it to the exchange's minimum notional requirement, $\text{min\_usd}$,
derived from the \texttt{hl\_meta} table.

\begin{table}[h]
  \centering
  \caption{Missing Positions by Root Cause (Post--Dec 2)}
  \begin{tabular}{lrr}
    \toprule
    Reason              & Count & Share of Missing (\%) \\
    \midrule
    BELOW\_MIN\_NOTIONAL & 72   & 86 \\
    ABOVE\_MIN\_NOTIONAL & 12   & 14 \\
    \bottomrule
  \end{tabular}
\end{table}

Relative to \emph{all} 182 symbol--days:
\begin{itemize}
  \item $\approx 39.6\%$ are missing because target notional is below
    the exchange's minimum (structural constraint for a \$2000 book).
  \item $\approx 6.6\%$ are missing despite being above the minimum---these are
    genuine execution or infra failures.
\end{itemize}

\section{Wrong-Sign Positions and Day-Over-Day Flips}

WRONG\_SIGN cases are those where the actual exposure direction
disagrees with the aligned T+2 signal, with non-trivial magnitude.
For each such symbol--day we re-check the sign of the model at
neighboring offsets:
\begin{itemize}
  \item $T+1$ (using signal from $T-1$ in practice, ``T-1 offset''), and
  \item $T+3$ (delayed execution, ``T-3 timing'').
\end{itemize}

We classify each wrong-sign row as:
\begin{itemize}
  \item \textbf{DUE\_TO\_T1\_OFFSET} if the actual sign matches the T+1-aligned
    signal but not T+2.
  \item \textbf{DUE\_TO\_T3\_TIMING} if the actual sign matches the T+3-aligned
    signal but not T+2 and not T+1.
  \item \textbf{UNEXPLAINED} otherwise.
\end{itemize}

For the post--Dec 2 period (22 wrong-sign cases), the breakdown from the
existing analysis is:
\begin{itemize}
  \item $\approx 58\%$ DUE\_TO\_T1\_OFFSET,
  \item $\approx 42\%$ DUE\_TO\_T3\_TIMING,
  \item $0\%$ UNEXPLAINED.
\end{itemize}
Equivalently, about $13$ of 22 wrong-sign trades are explained by using the
wrong day (T-1), and about $9$ are consistent with a one-day delay (T-3).

As a share of all 182 symbol--days:
\begin{align*}
  \text{T1-driven wrong sign} &\approx \frac{13}{182} \approx 7.1\%, \\
  \text{T3-driven wrong sign} &\approx \frac{9}{182} \approx 4.9\%.
\end{align*}

Thus roughly $12\%$ of all post--Dec 2 symbol--days are wrong sign,
and nearly all of those are due to day-over-day flips combined with
misaligned execution timing.

\section{Magnitude Errors}

MAGNITUDE\_ERROR cases have the correct sign but differ from the target weight
by more than $2\%$ of portfolio value:
\[
|\text{target\_weight} - \text{actual\_weight}| > 0.02.
\]

In the post--Dec 2 period:
\begin{itemize}
  \item 36 out of 182 symbol--days ($\approx 19.8\%$) fall into MAGNITUDE\_ERROR.
  \item This rate is essentially unchanged from the pre--Dec 2 period ($\approx 21\%$),
    indicating that sizing issues (partial fills, price drift, lot constraints)
    are still a major component of tracking error.
\end{itemize}

\section{Why Post--Dec 2 Is Still Broken}

For the post--Dec 2 period:
\begin{itemize}
  \item Only $21.9\%$ of symbol--days match the model.
  \item About $39.6\%$ are structurally untradeable (BELOW\_MIN\_NOTIONAL).
  \item The remaining $\approx 38.5\%$ are execution or infra issues:
    \begin{itemize}
      \item $\sim 6.6\%$ missing above min notional,
      \item $\sim 12\%$ wrong sign (mostly T-1 or T-3 timing),
      \item $\sim 20\%$ magnitude errors.
    \end{itemize}
\end{itemize}

In other words, even after the offset ``fix'', roughly two-fifths of the
tracking error is due to mismatched timing and execution logic, and roughly
another two-fifths is baked in by running too granular a model on a \$2000
account under exchange minimums.

\end{document}

